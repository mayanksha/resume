\cvsection{Projects}
\begin{cventries}

  \cventry
  {Member, Team Robocon IIT Kanpur, Prof. Ashish Dutta}
		{ABU RoboCon 2018, Shuttlecock throwing Robots}
  {IIT Kanpur}
  {ongoing}
  {
    \begin{cvitems}
	\item Working on an Autonomous Robot capable of playing \textit{ném còn} (Vietnamese Game, literally, throwing Shuttlecock).
	\item Using an Intel Realsense Depth Camera for 3D image processing to detect the Shuttlecock.
	\item Attempted ball-tracking by means of CamShift and Optical Flow (surrounding restrictions, hence shifted to stereo imaging) in OpenCV.
	\item Collaborated in the designing and making of Manual Robot which hands Shuttlecocks to Autonomous Robot. 
    \end{cvitems}
  }
  \cventry
  {Association of Computing Activities}
				{Linux from Scratch}
  {IIT Kanpur}
  {2nd Semester}
  {
    \begin{cvitems}
    \item Developed a Student Search in bash which scraped data from Office Automation Portal of IITK. 
    \item Used the ZeromQ library in python to implement client server model to play Collatz Conjecture Game.
    \item Familiarised with Shell Utilities, Linux Architecture, Bootloaders, etc.
	\item Created a bootable (SysVInit based) bare-bones linux distribution, compiling packages from tarballs, building upon the host system kernel and finally compiling the kernel for new distro.
	\item Created Scripts in NodeJS and Javascript to fetch all the BLFS Packages along with their installation commands, storing them as JSON files. 
    \end{cvitems}
  }
  
  	\cventry
  {TA-201 Group Project}
	{The Automatic Stamper}
  {IIT Kanpur}
  {3rd Semester}
  {
    \begin{cvitems}
    \item Worked on a stamper, which if connected to a motor could stamp four times in one revolution.
	\item Used the Geneva Mechanism and Chain \& Sprocket Mechanism to drive the stamper.
    \item Used the mechanical processes of Casting, Brazing, Welding and Sheet Metal Forming to build it.
    \end{cvitems}
  }

\end{cventries}

%%% Local Variables:
%%% mode: latex
%%% End:
