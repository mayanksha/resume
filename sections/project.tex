\cvsection{Key Projects}
\begin{cventries}

  % \cventry
  % {Cable Braiding Machine, IIT Kanpur}
  % {Course Group Project (TA-202)}
  % {IIT Kanpur}
  % {}
  % {
    % \begin{cvitems}
		% \item Worked on a gear-driven Cable Braiding Machine to braid multiple threads simultaneously.
		% \item Used various Mechanical Processes, including Lathe to fabricate the required parts. 
	% \end{cvitems}
  % }

  \cventry
  {Course Project, under Prof. Biswabandan Panda}
  {Microarchitectural Side and Covert-Channel Attacks}
  {}
  {\href{https://github.com/mayanksha/clflush\_attack}{\faGithubSquare\acvHeaderIconSep GitHub}}
  {
    \begin{cvitems}
        % \item Working on Reverse Engineering L3 cache replacement policy for Intel's Kaby Lake processors.
        \item Successfully mounted a covert channel attack on a victim running some benign process on a different core, using the FLUSH + RELOAD attack on cache, \textbf{achieving ~65+\% accuracy}.
        \item Developed a side-channel attack on the GnuPG cryptographic library, to observe cache accesses to critical functions like Square, Reduce, Multiply to get the RSA Private Key.
    \end{cvitems}
  }

  \cventry
  {Course Project, under Prof. Pramod Subramanyam}
  {Cryptographically Secure Key-Value Store}
  {\href{https://github.com/mayanksha/crypt-key-value-store}{\faGithubSquare\acvHeaderIconSep GitHub}}
  {}
  {
    \begin{cvitems}
    \item Designed and implemented a secure key-value store (in Golang) with sharing semantics (between different users), under the assumption that datastore is malicious.
    \item Used a multi-level block structure (with encrypted metadata) to effeciently share/append file, with strong guarantees on time complexities of various operations (AppendFile, ShareFile, RevokeFile).
    \end{cvitems}
  }

  \cventry
  {Course Project, under Prof. Pramod Subramanyam}
  {Securing Zoobar Server}
  {\href{https://github.com/mayanksha/CS628}{\faGithubSquare\acvHeaderIconSep GitHub}}
  {}
  {
    \begin{cvitems}
    \item Exploited vulnerabilities using control hijacking techniques like buffer overflow \& return-to-libc attacks.
        \item Fixed security bugs and implemented privilege separation in the login service of Zoobar web server.
    \end{cvitems}
  }

  \cventry
  {Course Project, under Prof. Debadatta Mishra}
  {GemOS - Operating Systems}
  {\href{https://github.com/mayanksha/CS330}{\faGithubSquare\acvHeaderIconSep GitHub}}
  {}
  {
    \begin{cvitems}
    \item Implemented Multi-level paging, signals like SIGINT, SIGSEGV and SIGFPE and exception handlers like page-faults and divide-by-zero, on gem5 architectural simulator.
    \item Implemented system calls like expand, shrink, write, sleep, clone, etc. and added process scheduling (init/cloned processes) with round-robin scheduling policy in GemOS.
    \item Designed and implemented a scalable ext-2 like FUSE-based filesystem for GemOS.
    \end{cvitems}
  }

  \cventry
  {Course Project, under Prof. Arnab Bhattacharya}
  {Databases Project - Traffic Management System}
  {\href{https://github.com/mayanksha/cs315-traffic-mgmt/blob/master/report.pdf}{\faGithubSquare\acvHeaderIconSep GitHub}}
  {}
  {
    \begin{cvitems}
    \item Designed collection schemas of various entities to store/fetch information in an efficient manner from a MongoDB database (with Mongoose as the ODM).
    \item Developed a MEAN web-app to permit Role-Based-Access-Control (RBAC) for different stakeholders in the Traffic Mgmt. System.
    \end{cvitems}
  }

  \cventry
  {Course Project, under Prof. Sandeep Shukla}
  {A Blockchain based Voting System with Biometric Verification}
  {\href{https://github.com/mayanksha/blockchain/tree/master/Project}{\faGithubSquare\acvHeaderIconSep GitHub}}
  {}
  {
    \begin{cvitems}
    \item Designed an Ethereum based Voting System, with Biometric (Fingerprint) voter Verification, and experimented with Fuzzy-hashing on fingerprint minutiae data (obtained using a Fingerprint Reader).
    \item Developed a back-end for the system (with a fallback to LDAP credentials) and deployed it over the Ethereum Ropsten Test Network.
    \end{cvitems}
  }

  % \cventry
  % {Machine Learning Course Project, under Prof. Piyush Rai}
  % {Zero-shot Sketch Based Image Retrieval (SBIR)}
  % {\href{https://github.com/mayanksha/cs771/tree/master/Project}{\faGithubSquare\acvHeaderIconSep GitHub}}
  % {}
  % {
    % \begin{cvitems}
	% \item Implemented Principal Component Analysis \& tried extending it to zero-shot setting, to test how it generalizes to other unseen classes.
    % \item Used TU Berlin Sketch Dataset to learn various features features for comparison to hand-drawn sketches.
    % % \item Used canny edge detection to convert images to basic sketches in OpenCV.
	% % \item Modeled these images using tensorflow by taking hints and code snippets from various research papers.
    % \end{cvitems}
  % }

  \cventry
  {Association of Computing Activities (ACA, IITK CSE)}
  {Linux from Scratch}
  {\href{https://github.com/mayanksha/lfs}{\faGithubSquare\acvHeaderIconSep GitHub}}
  {}
  {
    \begin{cvitems}
    \item Developed a Student Search in bash which scraped data from Office Automation Portal of IITK.
    \item Used the ZeromQ library in python to implement client server model to play Collatz Conjecture Game.
    \item Created a bootable (SysVInit based) bare-bones linux distribution, compiling packages from tarballs, building upon the host system kernel and finally compiling the kernel for new distro.
    \end{cvitems}
  }

      % \cventry
  % {TA-201 Group Project}
	% {The Automatic Stamper}
  % {IIT Kanpur}
  % {}
  % {
    % \begin{cvitems}
    % \item Worked on a stamper, which if connected to a motor could stamp four times in one revolution.
	% \item Used the Geneva Mechanism and Chain \& Sprocket Mechanism to drive the stamper.
    % \item Used the mechanical processes of Casting, Brazing, Welding and Sheet Metal Forming to build it.
    % \end{cvitems}
  % }

\end{cventries}

%%% Local Variables:
%%% mode: latex
%%% End:
